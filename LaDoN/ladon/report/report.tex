\documentclass{article}
\usepackage[utf8]{inputenc}
\usepackage[english]{babel}
\usepackage{apacite}
\usepackage{graphicx}
\usepackage{mathptmx}
\usepackage[font = {small, it}]{caption}
\DeclareCaptionLabelFormat{cont}{#1~#2\alph{ContinuedFloat}}
\captionsetup[ContinuedFloat]{labelformat=cont}
%\usepackage{subcaption}
\usepackage{subfigure}
\usepackage{float}
\usepackage{fancyhdr}
\usepackage{listings}

\lstset{language=Python}

%% CHANGE MARGINS

\addtolength{\oddsidemargin}{-.875in}
\addtolength{\evensidemargin}{-.875in}
\addtolength{\textwidth}{1.75in}

\addtolength{\topmargin}{-.875in}
\addtolength{\textheight}{1.75in}

%% make header

\fancypagestyle{plain}{
\fancyhf{}
\rhead{Mikkel Werling (201706722)}
\lhead{The Life and Death of Social Networks}
\cfoot{\thepage}
}
\pagestyle{plain}

\setlength{\parindent}{2em}
\setlength{\parskip}{1em}
\renewcommand{\baselinestretch}{1.5}

\title{The Life and Death of Social Networks: A network formation model for opinion dynamics}
\author{Mikkel Werling, (201706722)}
\date{}
\begin{document}
\maketitle


\section{Theory}

\subsection{Network Formation}

\subsubsection{Social Networks}
Here the presentation of the important characteristics of social networks should be included and why they come about. 
These include the high transitivity, clustering, communities, and exponential degree distributions.

\subsubsection{Candidate Models}

A special focus on the Herding Friends (animal model) as well as the how random are random friends. Other papers could also be good here.

\subsubsection{The problems with current models}

Calibration with data and how to test models. Models with fixed networks won't cut it. 

\subsection{Social Influence}

Including some of the basic literature (Axelrod)

\subsubsection{Shaping opinions}

Introduce the evidence from psychology and computational literature to show why the assumptions in the model make sense

\subsubsection{Models of Social Influence}

Report the evolution of models and where to place this model in all of the literature

\subsection{A network formation model for social influence}

Explain the importance of making both a network formation and opinion dynamics model in one go

\section{Methods}

\subsection{Model specification}

Explain the different parameters of the model

\subsection{Model fitting}

explain how the model was calibrated (Bayesian Hyperparameter Optimization)

\section{Model investigation}

Get familiar with the different parameters and their interpretations

\subsection{The effect of randomness}

How randomness affects the distribution of opinions, as well as the network

\subsection{The effect of the boundary threshold}

How the boundary threshold affects the distribution of opinions, as well as the network

\subsection{The effect of homophily}

How homophily affects the distribution of opinions, as well as the network

\subsection{Important interactions}

Point to some of the important interactions (possible Golden zones)

\section{Results}

\subsection{Network generation}

\subsection{Opinion generation}

\section{Discussion}

\section{Conclusion}

\bibliographystyle{apacite}

\bibliography{../Referencer_new}

\end{document}